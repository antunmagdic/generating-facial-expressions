\section{Related Work}

Generating a particular person’s face with different facial expressions can be 
used in a variety of applications, including face recognition 
\cite{cao2018}, \cite{parkhi2015}, face verification 
\cite{sun2014}, \cite{taigman2014}, emotion prediction, 
expression database generation, facial expression augmentation and 
entertainment.
 
Generative Adversarial Networks (GANs) are a powerful class of generative models
based on game theory. A typical GAN optimization scheme consists in 
simultaneously training a generator network to produce realistic fake samples 
and a discriminator network trained to distinguish between real and fake data. 
This idea is embedded by the so-called adversarial loss \cite{pumarola2020}.
 
DCGANs are generative convolutional networks based off of GANs. Based on work 
\cite{suit}, DCGANs seem promising. Using them it is possible to reconstruct the
original image with great accuracy. However, it is also shown that the network 
has not learned to modify the image.
 
According to the article \cite{pumarola2020}, the GANimation (Anatomically 
Consistent Facial Animation) proposed additionally controlled generated 
expressions by Action Units labels, and allowed a continuous expression 
transformation. The authors introduced an attention-based generator to promote 
the robustness of their model for distracting backgrounds and illuminations.
 
In some related work \cite{chen2018} it was used approach to construct double 
encoder GAN. Double encoder GAN is used for facial expression synthesis to 
extract the latent vectors and conditional labels features of the real image.
 
Recent advances in GANs have shown impressive results for task of facial 
expression synthesis. The most successful architecture is StarGAN \cite{choi2018}, that 
conditions GANs' generation process with images of a specific domain, namely a 
set of images of persons sharing the same expression \cite{pumarola2018}. In our
work we have based precisely on this approach.
