\section{Introduction}

There have been several researches that tackled the problem of facial expression
recognition. However, even though the problem of facial expression generation is
more challenging than expected, it has been less investigated in the 
state-of-the-art. Being able to automatically animate the facial expression from
a single image would open the door to many new exciting applications in 
different areas, including the movie industry, photography technologies, 
fashion, e-commerce business etc. As Generative Adversarial Networks (GANs) have
become more successful and more prevalent, a big progress has been made in this 
task. The most successful architecture is StarGAN, which is able not only to 
synthesize novel expressions, but also to change other attributes of the face, 
such as age, hair color or gender. In this project, we will use StarGAN to 
generate facial expressions corresponding to different emotions. A GAN model in 
this project is developed to take an image of a person’s face and a desired 
emotion as inputs and as an output one has that person’s face with the required 
emotion applied, while personalized features of the face remain preserved.
